\documentclass{article}
\usepackage[top=30truemm,bottom=30truemm,left=20truemm,right=20truemm]{geometry}
\usepackage{amsmath,amsfonts,amssymb}
\usepackage[dvipdfmx]{graphicx}
\usepackage[dvipdfmx]{hyperref}
\usepackage{float}
\usepackage{longtable}
\usepackage{tikz}
\usepackage{pgfplots}
\usepackage {diagbox}
\usepackage{here}
\usepackage{url}
\pgfplotsset{compat=newest}
\newlength\figHeight
\newlength\figWidth
\usepackage{hyperref}
\usepackage{fancyhdr}
\usepackage{graphicx,color}
\usepackage{mathtools}
\usepackage{amsthm}
\usepackage{siunitx}
\theoremstyle{plain}
\newtheorem{theorem}{Theorem}
\newtheorem*{theorem*}{Theorem}

\theoremstyle{definition}
\newtheorem{definition}{Definition}


\theoremstyle{plain}
\newtheorem{note}{Note}
\newtheorem*{note*}{Note}
\numberwithin{equation}{section}
\numberwithin{theorem}{section}
\numberwithin{definition}{section}
\numberwithin{note}{section}
\hypersetup{%
setpagesize=false,
 bookmarksnumbered=true,%
 bookmarksopen=true,%
 colorlinks=false,%
 linkcolor=blue,
 citecolor=red,
}  


\pagestyle{fancyplain}
     \chead{群論}
     \lhead{数学の花} 
     \rhead{\thepage} 
     \lfoot{}
     \rfoot{} 
     \cfoot{\thepage}
\renewcommand\plainheadrulewidth{.4pt}% headrule on plain pages
\author{}
\title{群論}
\makeatletter
\renewcommand*\l@section{\@dottedtocline{1}{1.5em}{2.3em}}
\makeatother
\begin{document}
\maketitle
%\setcounter{tocdepth}{2}
%\setcounter{chapter}{0}

% 目次の見出しだけデザインを変える

\tableofcontents

\clearpage
%\renewcommand{\thesection}{\Roman{section}}

\section{二項演算の定義}
集合$S\neq \emptyset$ を考える。$\forall a,b\in S$ に対して$\exists c \in S$ を対応させる法則つまり直積$S\times S$から$S$への写像を2項演算(law of composition)と呼ぶ. 記号的には
\begin{align}
     S\times S & \to S              \\
     (a,b)     & \mapsto a\cdot b=c
\end{align}
と表現される.
\section{群の定義}
集合$G\neq \emptyset$について, 2項演算
\begin{align}
     G\times G & \to G              \\
     (a,b)     & \mapsto a\cdot b=c
\end{align}
が与えられていて, 次の3つの条件を満たすとき, $G$を群と呼ぶ.\\
\vskip\baselineskip
\noindent
G1. \emph{結合法則} (associative law)$\forall a,b,c \in G$について

\begin{align}
     a\cdot (b\cdot c)=(a\cdot b)\cdot c
\end{align}\\
G2. \emph{単位元の存在} ある$e\in G$が存在して, $\forall a\in G$について
\begin{align}
     e\cdot a=a\cdot e=a
\end{align}

この$e$をGの\emph{単位元}(unit element)という.\\
G3. \emph{逆元の存在} $\forall a\in G$についてある$b\in G$が存在して

\begin{align}
     a\cdot b=b\cdot a=e
\end{align}

となる. この$b$を$a$の\emph{逆元}(inverse element)とよび,$a^{-1}$と表記する.
\begin{definition}
     群$G$の元の個数を$\lvert G\rvert$と表記してGの位数とよぶ.
\end{definition}

\section{様々な概念}
二項演算$(G,\cdot),(G,+)$について
\begin{enumerate}
     \item $(G,\cdot)$が半群: G1つまり結合法則が成立するもの
     \item $(G,\cdot)$がモノイド: G1とG2つまり結合法則が成立し単位元が存在するもの.

     \item 群$(G,\cdot)$が可換群, アーベル群: 群に\emph{交換法則}を追加したもの. つまり, $\forall a,b\in G$について$a\cdot b=b\cdot a$が成立
     \item $(G,+,\cdot)$が環: $(G,+)$が可換群$(G,\cdot)$がモノイド
     \item $(G,+,\cdot)$が可換環: 環$(G,+,\cdot)$について$(G,\cdot)$で交換法則が成立する
     \item $(G,+,\cdot)$が整域: 可換環$(G,+,\cdot)$について$(G,\cdot)$が\emph{零因子}をもたない.
     \item $(G,+,\cdot)$が斜体: 環$(G,+,\cdot)$について$(G,+)$の単位元$0$以外の元が$(G,\cdot)$において\emph{単元}である. つまり$(G,\cdot)$についてG3逆元をもつ
     \item $(G,+,\cdot)$が体: $(G,+,\cdot)$が可換環かつ斜体
\end{enumerate}

\section{群の性質}
\begin{theorem}
     群$(G,\cdot)$に対して単位元はただ一つ存在
\end{theorem}

証明: $e,e'\in G$を$G$の単位元としたとき($e$を群の定義から保証された単位元$e$に対してもう1つの単位元$e'$を考える)

\begin{align}
     e=ee'=e'
\end{align}

が成立.
\begin{theorem}
     群$(G,\cdot)$に対して逆元はただ一つ存在
\end{theorem}


証明: $\forall a\in G$についてある$b,c\in G$を$a$の逆元とすると

\begin{align}
     b=be=bac=ec=c
\end{align}

が成立.

\begin{definition}
     群Gについて$\forall a,b\in G$に対して,$ab=ba$を満たすとき, $G$を\emph{可換群}(commutative group)または
     \emph{アーベル群}(abelian group)とよぶ.
\end{definition}
\begin{definition}
     群$G$に含まれる位数の数を$G$の\emph{位数}(order)とよび, $\lvert G\rvert$と表現する.
\end{definition}
\begin{definition}
     $G$の位数が有限のとき$G$を\emph{有限群}(finite group), 無限のときは$G$を\emph{無限群}とよぶ.
\end{definition}
\section{部分群}
\begin{definition}\label{def::subgroup}
     群$G$の空でない部分集合$H$が$G$の二項演算によって群になるとき, $H$を$G$の\emph{部分群}(sub group)とよぶ.
\end{definition}
\begin{note}
     $G$の二項演算によって群になるとは演算$\cdot': H\times H\to H$が存在して, $\forall a,b\in H$について$a\cdot b=a\cdot' b$(よく似ていることの言い換え)となり, $(H,\cdot')$が群になること.
\end{note}
\begin{note}
     「群$(G,\cdot)$について$(H,\cdot)$が部分群であるとは$\emptyset\subsetneq H\subseteq G$かつ$(H,\cdot)$が群である.」は間違い.
\end{note}
(理由) $\cdot: G\times G\to G$であり, $\forall a,b\in H$について$a\cdot b\in H$は示されていない.\\

\par Definition~\ref{def::subgroup}を言い換たものが Definition~\ref{def::subgroup1}である.
\begin{definition}\label{def::subgroup1}
     群$(G,\cdot)$について$(H,\cdot )$が部分群であるとは以下すべてを満たすことである.
     \begin{enumerate}
          \item $\forall a,b\in H$に対して$a\cdot b\in H$.
          \item $\exists e \in H$, $\forall a \in H$に対して$a\cdot e=e\cdot a=a$.
          \item $\forall a \in H$, $\exists b\in H$に対して$a\cdot b=b\cdot a=e$.
          \item $\emptyset\subsetneq H\subset G$
     \end{enumerate}
\end{definition}
\begin{theorem}\label{thm::subgroup}
     $(G, \cdot)$を群とし, $\emptyset\subsetneq H\subseteq G$とする. このとき, $(H, \cdot)$が$G$の部分群であることは, 以下が両方成立することに同値:
     \begin{enumerate}
          \item[SG1.]  $\forall a, b\in G$. $a, b\in H\implies a\cdot b\in H$
          \item[SG2.]  $\forall a\in G$. $a\in H \implies a^{-1}\in H$
     \end{enumerate}
\end{theorem}
(証明)
$\implies$: $(H, \cdot)$が$G$の部分群であるとする.\\
\begin{itemize}
     \item SG1はDefnition~\ref{def::subgroup1}の1より成立.
     \item SG2はDefnition~\ref{def::subgroup1}の3より成立.
\end{itemize}
$\impliedby$: (SG1), (SG2)が成立しているとする.
\begin{itemize}
     \item 定義よりDefnition~\ref{def::subgroup1}の4が成立.
     \item  SG1よりDefnition~\ref{def::subgroup1}の1が成立.
     \item SG2よりDefnition~\ref{def::subgroup1}の3が成立.
     \item $\forall a\in H$についてSG1,SG2より$a^{-1}\in H$, $a\cdot a^{-1}=e$.
\end{itemize}
\begin{theorem}
     $H$を群$G$の部分群としたとき, $H$の単位元$e'$は$G$の単位元と一致する.
\end{theorem}
(証明) ある$x\in G$が存在して$e'x=xe'=e$になるとする. $e'=e'e=e'e'x=e'x=e$.
\begin{theorem}
     $H$を群$G$の部分群としたとき, $\forall a\in H$について$H$における逆元$a_H^{-1}$と$G$における逆元$a_G^{-1}$は一致する.
\end{theorem}
(証明) $a_G^{-1}=ea_G^{-1}=a_H^{-1}aa_G{-1}=a_H^{-1}e=a_H^{-1}$.
\begin{theorem}\label{them::subgroup3}
     群Gの部分集合$H\neq \emptyset$に対し, 以下の3条件は同値である.
     \begin{enumerate}
          \item $H$は$G$の部分群.
          \item $\forall a,b \in H$に対して$ab\in H$かつ$ a^{-1}\in H$.
          \item $\forall a,b \in H$に対して$a^{-1}\cdot b\in H$.
     \end{enumerate}
\end{theorem}
(証明) $1\implies 2$は部分群の定義から明らか. $2\implies 3$も明らか.
$2\implies 1$は Theorem~\ref{thm::subgroup}と同じである.
よって$3\implies 2$を示せばよい.
$\forall a\in H$について$e=a^{-1}\cdot a\in H$が成立. ここで$a^{-1}\in G$は$a^{-1}\cdot a^{-1}\cdot a=a^{-1}\cdot e \in H$より$a^{-1}e=a^{-1}\in H$.
よって$\forall a,b \in H$について$a^{-1}\in H$となり$a\cdot b=(a^{-1})^{-1}\cdot b\in H$.
\begin{theorem}
     $H_1$,$H_2$を群$G$の部分群とするならば, $H_1\cap H_2$も$G$の部分群である.
\end{theorem}
(証明) $\forall a,b\in H_1\cap H_2$について $a\cdot b\in H_1$かつ$a\cdot b\in H_2$. よって$a\cdot b\in H_1\cap H_2$.
また$a^{-1}\in H_1$かつ$a^{-1}\in H_2$. よって$a^{-1}\in H_1\cap H_2$. したがって$H_1\in H_2$は$G$の部分群となる.
\begin{theorem}
     Gを群とし$\emptyset \subsetneq H\subseteq G$とする。このとき以下3条件は互いに同値
     \begin{enumerate}
          \item HはGの部分群
          \item $(HH\subseteq H)$かつ$H^{-1}\subseteq H$
          \item $H^{-1}H\subseteq H$
     \end{enumerate}
\end{theorem}
(証明) これは Theorem~\ref{them::subgroup3}の言い換えである.
\begin{definition}
     Gを群としSをGの部分集合とする。このときSを含む最小の部分群を$\langle S \rangle$と書く。これをSから生成される部分群と呼ぶ.
\end{definition}

\end{document}